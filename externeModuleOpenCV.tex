\subsection{OpenCV}  
\glqq OpenCV ist eine freie Programmbibliothek mit Algorithmen für die
Bildverarbeitung und maschinelles Sehen. Sie ist für die Programmiersprachen C,
C++, Python und Java geschrieben und steht als freie Software unter den
Bedingungen der BSD-Lizenz. \grqq

\subsubsection{Canny Edges Filter}
Die Funktion wird im Kamera-Modul eingesetzt, um aus dem Kamera-Stream eine
Stream nur mit Umrissen zu konvertieren. Dafür wird zwischen benachbarten Pixel
die Grauwert-Differenz als Steigung, im 2-dimensinalen der sogeannte Gradient,
bestimmt. Auf Grund der Intensität des Gradienten kann eine Stelle als Kante
klassifiert werden.\\
Es  werden folgende Argumente genutzt:\\
canny = cv.Canny(image, minVal, maxVal)

[cannyThresholds.png]

Die Argumente Nr. 2 und 3 setzen einen unteren und oberen Threshold. Der untere
Wert beschreibt eine Grenze für Gradienten, die kleiner sind und nicht als
Kanten erkannt werden sollen. Gradienten, die oberhalb des maxVal liegen, werden
als Kanten erkannt. Gradienten, die zwischen den beiden Werten liegen, werden,
wenn sie mit Gradienten ausserhalb der Grenzen verbunden sind, denen zugeordnet,
und können sowohl im Bild erscheinen als auch nicht. Sind sie nicht verbunden,
werden sie nicht dargestellt.
Für die Anwendung haben sich die Grenzen von 80 und 140 als passend herausgestellt.

\subsubsection{Hough Space Transformation}
... hier weiter
