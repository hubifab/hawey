\section{Einleitung}
Im Jahr 2005 ging das Projektauto Stanley der Standford University an den Start
der DARPA Grand Challenge, einem Rennen autonomer Fahrezeuge über eine
definierte Strecke in der Mojave-Wüste. Nach 212,76km durch unwegsames Gelände
ging Stanley als Erster über die Ziellinie und das Team um Professor Sebastian
Thrun war Gwinner des mit 2 Millionen Dollar dotierten Rennens.\\
Die Entwicklung des Autonomen Fahrens für Fahrzeuge des Straßenverkehrs ist ein
aktuelles Thema, das immer mehr in die Praxis umgesetzt wird. Dabei gibt es
verschiedenen Stufen des teilautonomen Fahrens. So gehört zur Stufe 2 die
Funktion, dass Lenkvorgänge vom Fahrassistenten ausgeführt werden.\\
Durch dieses Projekts soll ein Einblick in die Materie geschaffen
werden und die Umsetzbarkeit im Rahmen studentischer Möglichkeiten erfahren
werden .\\
Die Autoren wollen mit ihrem Auto Hawey einen Startschuß setzen und mit diesem
Bericht ein nachfolgendes Team inhaltlich in der Sache abholen, dass ein
Einstieg in das Projekt und so ein konstruktives Fortsetzten und Verbessern 
möglich ist.\\

Fabian Huber, Enzo Morino, Markus Trockel
