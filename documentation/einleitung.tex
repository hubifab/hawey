\section{Einleitung}
Im Jahr 2005 ging das Projektauto Stanley der Standford University an den Start
der DARPA Grand Challenge, einem Rennen autonomer Fahrezeuge über eine
definierte Strecke in der Mojave-Wüste. Nach 212,76km durch unwegsames Gelände
ging Stanley als Erster über die Ziellinie und das Team um Professor Sebastian
Thrun war Gwinner des mit 2 Millionen Dollar dotierten Rennens.\\
Die Entlung des Autonomen Fahrens für Fahrzeuge des Straßenverkehrs ist ein
aktuelles Thema, das immer mehr in die Praxis umgesetzt wird. Dabei sind
verschiedenen Stufen des teilautonomen Fahrens. So gehört zur Stufe 2 die
Funktion, Lenkvorgänge vom Fahrassistenten ausgeführt werden.\\
Um einen Einblick in die Materie zu bekommen und die Umsetzbarkeit im Rahmen eines 
Modellwagens zu erörten, ist dieses Projekt estartet worden. Durch die Realisierung 
wurden sowohl zum einen die verfügbaren Umsetzungsmöglichkeiten erfahrbar, wie aber 
auch unerwartete Herasuforderungen, für die eine Lösungen gefunden werden muss.\\
Die Autoren wollen mit ihrem Auto Hawey einen Startschuß setzen und mit diesem
Bericht ein nachfolgendes Team inhaltlich in der Sache so abholen, dass ein
Einstieg in das Projekt und ein konstruktives Fortsetzten und weiteres Verbessern 
möglich ist.\\

Fabian Huber, Enzo Morino, Markus Trockel
