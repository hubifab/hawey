\subsection{Kamera}
Die Kamera soll ein Bild der Straße einfangen, um daraus die Fahrbahnlininen zu
erkennen. 
\subsubsection{Position}
Kamerabilder von recherchierten Projekten, insbesondere bei
Teilnehmern des Carolo-Cups, zeigen eine Position, die 
erhöht am hinteren Ende des Wagens angebracht liegt. Diese Position kann sehr
unterschiedliche Lagen der Fahrbahnlinien im Bild erzeugen, in Bezug auf
erkannter Winkel und Offset an der linken und rechten Y-Achse des Bildes.
Möglicherweise wird hier eine Auswertung der Information verfolgt, bei der eine
Transformation in eine senkrechte Ansicht auf die Straße gemacht wird, um dann
die Steuerinformaiton zu generieren.\\
Für das Projekt ist eine Bildauswertung gewählt worden, bei der ein vertikal
symmetrisches Bild erzeugt wird, so dass sowohl Winkel als auch Offset der
Fahrbanlinien nicht zu extreme Werte annehmen können. Dafür ist die Kamera an
der Front des Wagens in Stoßstangenhöhe positioniert worden. So ist eine
Berechnung des Fluchtpunkts der Linien stabiler und besser für eine
Richtungsgebung auswertbar.\\
