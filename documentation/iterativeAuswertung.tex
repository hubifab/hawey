\section{Iterative Auswertung des Kamerabildes}	

	Das Bild der Kamera werden iterativ ausgewertet um die beiden Fahrbahnlinien zu erkennen und daraus ein Ausgangssignal für die Steuerung zu generieren. Der gewählte Ansatz war es, die Fahrbahnlinien durch zwei Geraden anzunähern. Die Funktion HoughLines wird verwendet um die Fahrbahnlinien zu erkennen und die zugehörigen Geraden zu berechnen. Da die Funktion HoughLines in den meisten Fällen nicht nur die beiden Fahrbahnlinien erkennt, sondern alle möglichen Geraden, bestand die erste Aufgabe darin, nur die Geraden herauszufiltern, welche die Fahrbahnlinien repräsentieren, und alle anderen Geraden zu verwerfen. Im Programm wird dies erreicht, indem alle Geraden, die durch die Funktion HoughLines gefunden wurden mit dem Geradenpaar aus dem vorherigen Programmdurchlauf verglichen werden, und nur ähnliche Geraden beibehalten werden.\\
	
	Beim Start des Programm werden für die beiden Geraden feste Werte vorgegeben, die dann korrigiert und an die erkannten Fahrbahnlinien angepasst werden. In Abbildung \ref{fig:fahrbahn} ist dies bildlich dargestellt. Die roten Linien sind zu Beginn des Programms fest vorgegeben. Die blauen Linien sind das Ergebnis der Korrektur durch das Programm.
	
	\begin{figure}[H]
		\centering
		\includegraphics[width=.5\linewidth]{images/fahrbahn.png}
		\caption{Korrektur der vorgegeben Geraden}
		\label{fig:fahrbahn}
	\end{figure}
	
	
	Der Vergleich der Geraden wird durchgeführt, indem um die beiden Geraden aus dem letzten Durchlauf ein Toleranz-Fenster gelegt wird, und von allen gefundenen Geraden nur diejenigen herausgefiltert werden, die innerhalb dieses Toleranz-Fensters liegen. Von diesen Geraden wird dann der Durschnitt berechnet, sodass nur noch zwei Geraden für die beiden Fahrbahnlinien übrig bleiben. \\
	
	Eine Schwierigkeit besteht darin die Geraden so zu repräsentieren, dass diese gut miteinander vergleichbar sind. Bei einer geringen Änderung einer Geraden von einem zum nächsten Zyklus sollen sich deren Parameter dabei auch nur geringfügig ändern. Ansonsten werden ähnliche Geraden beim filtern verworfen, da deren Parameter nicht im Toleranzfenster liegen. Im Programm wurden daher je nach Anwendungsfall verschiedene Repräsentationsformen verwendet, welche im folgenden beschrieben werden.
	
	\subsection{Verwendete Repräsentationsformen von Geraden}
	
	Die Funktion HoughLines gibt als Ergebnis eine Liste von Geraden aus, die durch die beiden Parameter $\rho$ und $\theta$ beschrieben werden. $\theta$ ist der Winkel zwischen der Normalen, die senkrecht auf der Geraden steht, und der x-Achse. Der Winkel  $\rho$ ist die Länge dieser Normalen, d.h. der Abstand zwischen der Geraden und dem Ursprung des Koordinatensystems. Der Winkel $\theta$ ist immer positiv, $\rho$ kann, je nach Lage der Geraden positiv oder negativ sein.
	Abbildung \ref{fig:rho_theta1} zeigt ein Beispiel für die Repräsentation zweier Fahrbahnlinien durch einen Abstand $\rho$ und einen Winkel $\theta$, für den Fall, dass der Wagen relativ mittig und gerade auf der Fahrbahn steht. Dabei symbolisiert das Rechteck die Ränder des Kamerabildes, der Koordinatenursprung liegt in der oberen linken Ecke. $\theta_R$ und $\rho_R$ sind die Parameter der rechten Fahrbahnlinie, $\theta_L$ und $\rho_L$ die der Linken.
	
	\begin{figure}[H]
		\centering
		\includegraphics[width=.5\linewidth]{images/rho_theta1.jpg}
		\caption{Beispiel für die Geradenrepräsentation durch einen Winkel $\theta$ und einen Radius $\rho$}
		\label{fig:rho_theta1}
	\end{figure}
	
	Diese Form der Geradenrepräsentation ist jedoch problematisch, wenn es darum geht, zwei Geraden miteinander zu vergleichen. Ein Beispiel ist in Abbildung \ref{fig:rho_theta2} dargestellt. Hier beträgt die Änderung der Winkel $\theta_R$ und $\theta_L$ der beiden Fahrbahnlinien $\Delta\theta=5^\circ$, obwohl die Position der rechten Fahrbahnlinie sich viel mehr verändert hat, als die der linken.
	
	
	
	\begin{figure}[H]
		\centering
		\includegraphics[width=.5\linewidth]{images/rho_theta2.jpg}
		\caption{Problematik bei dieser Form der Geradenrepräsentation}
		\label{fig:rho_theta2}
	\end{figure}

	Um dieses Problem zu beheben wurden die Geraden mithilfe der Funktion get\_offset\_alpha1 in eine andere Form umgerechnet, bei der diese durch den Offset $\omega$ auf der y-Achse und den Winkel zur x-Achse beschrieben werden. Die Umrechnung von der $\rho$-$\theta$-Form erfolgt folgendermaßen:\\

	
	
	Für Winkel $\theta<90^\circ$ (linke Fahrbahnlinie):
	
	\begin{align*}
	\alpha=90^{\circ}-\theta
	\omega&=-\frac{\rho}{\cos{90^{\circ}-\theta}} \\
	\end{align*}
	
	Für Winkel $\theta>90^\circ$ (rechte Fahrbahnlinie):
	
	\begin{align*}
	\alpha=\theta-90^{\circ}
	\omega&=-\frac{\rho}{\cos{\theta-90^{\circ}}} \\
	\end{align*}
	
	Abbildung \ref{fig:alpha_omega1} veranschaulicht die Repräsentation der beiden Fahrbahnlinien in dieser Form.
	
	\begin{figure}[H]
		\centering
		\includegraphics[width=.5\linewidth]{images/alpha_omega1.jpg}
		\caption{Beispiel für die Geradenrepräsentation durch einen Offset $\omega$ und einen Winkel $\alpha$}
		\label{fig:alpha_omega1}
	\end{figure}

	Bei dieser Form ergibt sich jedoch das Problem, dass sich bei steilen Geraden der Offset stark ändert, auch wenn der Winkel sich nur leicht ändert. Dadurch werden diese Geraden dann ungewollt verworfen. Dies ist besonders für die rechte Fahrbahnlinie der Fall, wie sich in Abbildung \ref{fig:alpha_omega2} erkennen lässt. Hier sieht man, dass sich der Offset $\omega$ stark ändert, auch wenn sich der Winkel der Geraden nur leicht ändert.


	\begin{figure}[H]
		\centering
		\includegraphics[width=.3\linewidth]{images/alpha_omega2.jpg}
		\caption{Problematik bei dieser Form der Geradenrepräsentation}
		\label{fig:alpha_omega2}
	\end{figure}

	Daher wurden für die beiden Geraden zwei verschiedene Repräsentationsformen gewählt. Für die linke ist der Offset als Schnittpunkt mit dem Linken Bildrand definiert, für die rechte Gerade ist es der Schnittpunkt mit dem rechten Bildrand. Die Winkel werden wie gehabt zur x-Achse gemessen. Die Umformung von der $\rho$-$\theta$-Form geschieht in der Funktion get\_offset\_alpha2 durch folgende Berechnungen: \\
	
	\begin{align*}
	\alpha=90^{\circ}-\theta
	\end{align*}
	
	
	Für Winkel $\theta<90^\circ$ (linke Fahrbahnlinie):
	
	\begin{align*}
	\omega&=-\frac{\rho}{\cos{90^{\circ}-\theta}} \\
	\end{align*}
	
	Für Winkel $\theta>90^\circ$ (rechte Fahrbahnlinie):
	
	\begin{align*}
	\omega&=\frac{B+\frac{\rho}{\cos{\theta-90^{\circ}}}}{\tan{90^{\circ}-\theta}} \\
	\end{align*}
	
	Mit der Bildbreite B.
	
	Abbildung \ref{fig:alpha_omega3} veranschaulicht diese Repräsentationsform.
	
	\begin{figure}[H]
		\centering
		\includegraphics[width=.5\linewidth]{images/alpha_omega3.jpg}
		\caption{Beispiel für die Geradenrepräsentation durch Winkel $\alpha$ und Offsets $\omega$ gemessen von der linken und rechten oberen Ecke}
		\label{fig:alpha_omega3}
	\end{figure}


	Auch bei dieser Repräsentationsform könnten sich Problematische Fälle ergeben. Es wurde jedoch festgestellt, dass diese Form bisher am besten funktioniert.\\
	
	Alternativ wäre es möglich, die Geradenfindung nicht iterativ durchzuführen, und somit keinen Vergleich zwischen alten und neuen Gefundenen geraden durchführen zu müssen. Dazu müsste die Kamera so steil nach unten ausgerichtet werden, dass garantiert wird, dass nur die beiden Fahrbahnlinien erkannt werden, und keine anderen Linien.


	\subsection{Berechnung des Schnittpunktes der beiden Geraden}
	
	Der Lenkwinkel wird mithilfe der x-Koordinate des Fluchtpunktes, also des Schnittpunktes der beiden Geraden berechnet. Die Abweichung dieses x-Wertes vom Mittelpunkt wurde als Regeldifferenz verwendet.\\
	Die x-Koordinate des Fluchtpunktes (engl. vanishing point) wird in der Funktion get\_vp mit folgender Formel berechnet:\\
	
	\begin{align*}
	vp_x=\frac{\omega_R-\omega_L}{\tan{\alpha_L}+\tan{\alpha_R}} \\
	\end{align*}
	
	Um die Abweichung vom x-Wert der Bildmitte zu erhalten, wird von diesem Wert die Hälfte der Bildbreite B abgezogen:\\
	
	\begin{align*}
	\Delta x=vp_x-\frac{B}{2} \\
	\end{align*}
	
	
	\begin{figure}[H]
		\centering
		\includegraphics[width=.5\linewidth]{images/vp.jpg}
		\caption{Regeldifferenz als Differenz zwischen Bildmitte und x-Koordinate des Fluchtpunktes}
		\label{fig:alpha_omega3}
	\end{figure}

	Im Nachhinein wurde jedoch festgestellt, dass die Lage des Fluchtpunktes allein nicht ausreichend für die Steuerung des Wagen ist, da diese durch zwei Variablen definiert wird: Die horizontale Lage des Wagens auf der Fahrbahn (links, rechts) und seine Ausrichtung. So kann der Fluchtpunkt z.B. nach links verschoben sein, wenn der Wagen nah an der linken Fahrbahnlinie ist aber geradeaus fährt, oder aber wenn er mittig auf der Fahrbahn ist, aber nach rechts fährt. In beiden Situationen ist jedoch genau das umgekehrte Verhalten notwendig.
	
	\begin{itemize}
		\item Wagen nah an linker Fahrbahnlinie: nach rechts lenken
		\item Wagen mittig aber nach rechts ausgerichtet: nach links lenken
	\end{itemize}

	Es sind also noch mehr Kriterien einzubeziehen um den Wagen erfolgreich auf der Fahrbahn zu halten. Dabei besteht die Schwierigkeit darin, wie die verschiedenen Informationen (z.B. die beiden Winkel und Offsets der Linien) miteinander verrechnet werden, um am Ende eine Größe zu erhalten, die als Regelgröße verwendet werden kann. 
	
	
	
	