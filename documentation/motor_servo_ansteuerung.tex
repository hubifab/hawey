\section{Ansteuerung von Servo und Motorcontroller}


Motorcontroller und Servo werden mithilfe eines Adafruit PCA9685 16-Kanal Servo Treibers angesteuert. Dieser nimmt Befehle des Raspberry Pi's über $I^2C$ entgegen und wandelt sie in ein PWM-Signal um. Das PWM Signal hat eine Periodendauer von 20ms, die Einschaltdauer $T_{on}$ liegt zwischen 1ms und 2ms. \\
Der Servotreiber hat eine eigene Bibliothek aus der am wichtigsten die Funktion set\_pwm(kanal, on,off) ist. Als Kanal wurde für den Servo 0 gewählt, für den Motorcontroller 1. Der Einschaltzeitpunt ''on'' wird immer auf 0 gesetzt. Für den Auschaltzeitpunkt kann theoretisch eine Zahl zwischen 0 und 4095 gesetzt werden. Da beim Servo das On-Signal zwischen einer und zwei Millisekunden lang sein muss ergibt sich der theoretische Wert folgendermaßen: \\

\begin{align*}
	off&=T_{on}\cdot\frac{20ms}{4096}\\
	off_{min}&=1ms\cdot\frac{20ms}{4096} =204,8\\
	off_{max}&=2ms\cdot\frac{20ms}{4096} = 409,6
\end{align*}

Für das verwendete Auto wurde der Min- und Maxwert empirisch ermittelt indem die geschaut wurde, bei welchen Werten die Räder bis zum Anschlag ausgelenkt sind. Es wurden folgende Werte ermittelt:\\
Vollausschlag rechts: 272\\
Vollausschlag rechts: 342\\
Mittelposition: 307\\

Für den Vollausschlag nach rechts und links wurden die Werte 297 und 317 Verwendet, damit nicht zu scharfe Kurven gefahren werden.\\

Ebenso wurden für den Motor die Werte für den off-Parameter durch Ausprobieren ermittelt, indem geschaut wurde, bei welchen Werten das Auto mit angemessener Geschwindigkeit fährt. Hier wurden ermittelt:\\
Vorwärts: 320\\
Rückwärts: 295\\
Stop: 307\\


