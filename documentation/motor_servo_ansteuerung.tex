\section{Ansteuerung von Servo und Motorcontroller}


Motorcontroller und Servo werden mithilfe eines Adafruit PCA9685 16-Kanal Servo Treibers angesteuert. Dieser nimmt Befehle des Raspberry Pi's über $I^2C$ entgegen und wandelt sie in ein PWM-Signal um. Das PWM Signal hat eine Periodendauer von 20ms, die Einschaltdauer $T_{on}$ liegt zwischen 1ms und 2ms. \\
Der Servotreiber hat eine eigene Bibliothek aus der am wichtigsten die Funktion set\_pwm(kanal, on,off) ist. Als Kanal wurde für den Servo 0 gewählt, für den Motorcontroller 1. Der Einschaltzeitpunt ''on'' wird immer auf 0 gesetzt. Für den Auschaltzeitpunkt kann theoretisch eine Zahl zwischen 0 und 4095 gesetzt werden. Da beim Servo das On-Signal zwischen einer und zwei Millisekunden lang sein muss ergibt sich der theoretische Wert folgendermaßen: \\

\begin{align*}
	off&=T_{on}\cdot\frac{20ms}{4096}\\
	off_{min}&=1ms\cdot\frac{20ms}{4096} =204,8\\
	off_{max}&=2ms\cdot\frac{20ms}{4096} = 409,6
\end{align*}

Für das verwendete Auto wurde der Min- und Maxwert empirisch ermittelt indem die geschaut wurde, bei welchen Werten die Räder bis zum Anschlag ausgelenkt sind. Es wurden folgende Werte ermittelt:\\

\begin{itemize}
	\item Vollausschlag rechts: 272
	\item Vollausschlag rechts: 342
	\item Mittelposition: 307
\end{itemize}

Als geeigneter Ausschlag nach rechts und links wurden die Werte 297 und 317 verwendet, damit nicht zu scharfe gelenkt wird.\\

Ebenso wurden für den Motor die Werte für den off-Parameter durch Ausprobieren ermittelt, indem geschaut wurde, bei welchen Werten das Auto mit angemessener Geschwindigkeit fährt. Hier wurden ermittelt:\\

\begin{itemize}
	\item Vorwärts: 320
	\item Rückwärts: 295
	\item Stop: 307
\end{itemize}


Die Funktion sendCommand(command) aus dem Modul modAct.py nimmt eine einen ganzzahligen Wert ''command'' entgegen, und sendet daraufhin mit set\_pwm ein PWM-Signal an den Servo oder Motorcontroller. Für vorwärts, rückwärts, stop, links, rechts und mitte wurden verschiedene Werte vorbelegt, mit bei denen set\_pwm mit den oben aufgelisteten Parametern aufgerufen wird. In diesen Fällen funktioniert die Steuerung diskret und es können nur drei feste Winkel und drei feste Motordrehzahlen eingestellt werden.\\
Ebenso kann sendCommand mit der einer Regeldifferenz, z.B. der Verschiebung des Fluchpunktes in x-Richtung, aufgerufen werden. Diese wird im aktuellen Programm direkt als Parameter der Funktion set\_pwm übergeben. In zukünftigen Projekten könnte man diese aber auch mit einer Konstanten multiplizieren, integrieren oder differenzieren um einen P-, PI- oder PID Regler zu implementieren.


