\section{Software}
  \subsection{Einrichtung der Netzwerkverbindung}
  Die Steuerungssoftware läuft vollständig auf dem Raspberry Pi. Um komfortabel
  am Quellcode arbeiten zu können ohne jedesmal Monitor und Tastatur anschließen
  zu müssen, wird der Wifi-Zugang eingerichtet und eine Backup-Verbindung über
  ein Ethernetkabel konfiguriert. \\

  \subsubsection{Wifi}
    Zum aktivieren der Wifi-Verbindung
    kann einfach das Tool \texttt{raspi-config} verwendet werden. Gestartet
    werden kann das Tool durch Eingabe des folgenden Befehls im
    Terminal-Fenster: \\
    \ \\
    \texttt{sudo raspi-config} \\
    \ \\
    Im Unterpunkt \textbf{Network Options} findet sich die Funktion
    \textbf{Wifi}. Im Folgenden wird der Benutzer zur Eingabe des Netzwerknamen
    (SSID) und dem Passwort aufgefordert. \\
    \ \\
    Um die Netzwerkeinstellungen händisch hinzuzufügen reicht es in der Regel, die
    Datei \texttt{/etc/wpa\_supplicant/wpa\_supplicant.conf} mit einem Editor zu
    öffnen (Achtung: Root-Rechte erforderlich) und folgenden Eintrag am Ende der
    Datei anzufügen:

    \begin{lstlisting}
    network={
      ssid="NameDesNetzwerks"
      psk="Passwort"
    }
    \end{lstlisting}
    \vspace{0.5cm}

    \noindent
    Nach der Änderung der Einstellung kann der Netzwerk-Service neu gestartet
    werden: \\
    \ \\
    \texttt{sudo service networking restart} \\
    \ \\

  \subsubsection{Ethernet}
    Gelegentlich kann es vorkommen, dass sich der Raspberry Pi nicht automatisch
    mit dem eingerichteten Wifi verbindet, zum Beispiel wenn das WLAN Netzwerk
    nicht verfügbar ist. Um sich in dieser Situation trotzdem mit dem Gerät
    verbinden zu könnne, wird eine Backup-Verbindung über Ethernet-Kabel
    eingerichtet. Damit kann sich der Benutzer bei Bedarf per Kabel mit seinem
    Laptop direkt mit dem Raspberry Pi verbinden. \\
    \ \\
    Hierfür kann mit einem Editor die Konfigurationsdatei
    \texttt{/etc/network/interfaces} geöffnet werden. Für das entsprechende
    Ethernet-Gerät, in diesem Fall \texttt{eth0}, wird somit eine statische IP
    Adresse vergeben, über die dann künftig ebenfalls eine SSH-Verbindung
    aufgebaut werden kann.

    \begin{lstlisting}
    iface eth0 inet static
    address 192.168.1.10
    netmask 255.255.255.0
    gateway 192.168.1.1
    \end{lstlisting}
    \vspace{0.5cm}

  \subsubsection{Login per SSH}
    Nachdem das Netzwerk eingerichtet ist und eine Verbindung zum Wifi Netzwerk
    steht, wird zum Login über das Netzwerk die neue IP Adresse des Raspberry Pi
    benötigt. Um diese herauszufinden gibt es zahlreiche Möglichkeiten. Eine
    davon ist direkt am Raspberry Pi im Terminal das Kommando \texttt{ifconfig}
    einzugeben und zu bestätigen. In der Ausgabe findet sich unter der
    Bezeichnung des Wifi-Geräts, meinst \texttt{wlan0}, ein Eintrag \texttt{inet
    addr}, gefolgt von einer IP Adresse. Mit Hilfe dieser Adresse lässt
    sich nun von jedem Computer im selben Wifi Netzwerk eine Verbindun zum
    Raspberry Pi herstellen (die 'x' stehen hier für die eigentliche IP Adresse): \\
    \ \\
    \texttt{ssh -X pi@xxx.xxx.xxx.xxx} \\
    \ \\
    Standard Benutzername ist hier 'pi' und das Passwort nach dem gefragt wird
    ist 'raspberry'. Der Parameter \texttt{-X} ermöglicht die grafische Ausgabe
    des Programms ebenfalls an die lokale Ausgabe des entfernten Laptops
    weiterzuleiten.\\


  \subsection{Aufbau}
  \ \\
  \begin{minipage}{\columnwidth}
    \makeatletter
    \def\@captype{figure}
    \makeatother
    \centering
    \includegraphics[width=0.8\linewidth]{images/code-flowchart.png}
    \caption{Aufbau des Python Codes}
    \label{fig:image-01}
  \end{minipage}
  \ \\

  \subsection{Externe Module}
  \ \\
  \begin{minipage}{\columnwidth}
    \makeatletter
    \def\@captype{table}
    \makeatother
    \centering
    %\rowcolors{1}{grey}{white}
    \begin{tabular}{p{3.5cm}|p{11cm}}
    % \multicolumn{2}{|c}{Frame \#} & \multicolumn{4}{|c}{LCD 0/3} &
    Name & Beschreibung \\ \hline \hline
      & \\
    Adafruit\_PCA9685 & Bibliothek zur Ansteuerung des Motorcontrollers \\
      & \\
    numpy & Bibliothek zur Verwendung von Matlab Funktionen \\
      & \\
    cv2 & OpenCV 2 bietet Algorithmen zur Bildverarbeitung \\
      & \\
    io & Bibliothek zur Behandlung und Verarbeitung unterschiedlicher I/O Streams\\
      & \\
    time & Bibliothek zur Behandlung und Umwandlung von Zeitstempeln \\
      & \\
    importlib & Ermöglicht die Benutzung der 'import' Funktion \\
      & \\
    argparse & Ermöglicht die einfache Behandlung von übergebenen Argumenten\\
      & \\
    pivideostream & Teilpacket der Bibliothek \texttt{imutils}\\
      & \\
    picamera & Bibliothek zu direkten Interaktion mit dem Pi Camera Modul \\
      & \\
    threading & Stellt High Level Funktionen zur Parallelverarbeitung mit Threading zur Verfügung \\
      & \\
    RPi.GPIO & Bibliothek zur Ansteuerung der GPIO ports des Raspbery Pi \\
    \end{tabular}
    \caption{verwendete externe Python Module}
    \label{tab:01}
  \end{minipage}
  
  \subsection{Eigene Module}
  \ \\
  \begin{minipage}{\columnwidth}
    \makeatletter
    \def\@captype{table}
    \makeatother
    \centering
    %\rowcolors{1}{grey}{white}
    \begin{tabular}{p{3.5cm}|p{11cm}}
    % \multicolumn{2}{|c}{Frame \#} & \multicolumn{4}{|c}{LCD 0/3} &
    Name & Beschreibung \\ \hline \hline
    modAnalysis & Verantwortlich für die eigentliche Verarbeitung der visuellen
      Informationen. \\
    modAct & Verantwortlich für die Ansteuerung des Motors und der Lenkung. \\
    modCamera & Bereitet das Kamerabild für die Verarbeitung und Anzeige vor. \\
    modSonic & Kommuniziert mit dem Ultraschallsensor und liefert Distanz zum Hindernis.\\
    \end{tabular}
    \caption{verwendete eigene Python Module}
    \label{tab:01}
  \end{minipage}

