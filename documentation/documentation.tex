\documentclass[a4paper,12pt]{article}

% include headers and preamble for reoport
% file: includes-report.tex
% -----------------------------------------------------------------------------
% includes for studies report
% -----------------------------------------------------------------------------

\usepackage{amsmath}
\usepackage{setspace}
\usepackage[top=1.2in, bottom=1.2in]{geometry}
\usepackage[x11names]{xcolor}

\usepackage{graphicx}
\usepackage[utf8]{inputenc}
\usepackage{siunitx}

%\usepackage{multirow}
%\usepackage{pgfplots}
\usepackage{subcaption}

% // -- for source code listings --
\usepackage{color}
\usepackage{xcolor}
\definecolor{OliveGreen}{RGB}{0,128,0}
\usepackage{listings}
\usepackage{caption}
\DeclareCaptionFont{white}{\color{white}}
\DeclareCaptionFormat{listing}{\colorbox{gray}{\parbox{\textwidth}{#1#2#3}}}
\captionsetup[lstlisting]{format=listing,labelfont=white,textfont=white}


\lstdefinestyle{cStyle}{language=C}
\lstset{
language=C,
%basicstyle=\small\ttfamily,
basicstyle=\small\ttfamily,
keywordstyle=\color{blue}\ttfamily,
stringstyle=\color{red}\ttfamily,
commentstyle=\color{magenta}\ttfamily,
morecomment=[l][\color{magenta}]{\#},
numbers=left,
numberstyle=\tiny,
% frame=tb,
columns=fullflexible,
showstringspaces=false,
tabsize=2
}
\usepackage{matlab-prettifier}
\lstdefinestyle{matlabStyle}{language=matlab}
\lstset{
%style=Matlab-editor,
language=matlab,
%basicstyle=\small\ttfamily,
basicstyle=\small\ttfamily,
keywordstyle=\color{blue}\ttfamily,
stringstyle=\color{red}\ttfamily,
commentstyle=\color{OliveGreen}\ttfamily,
morecomment=[l][\color{OliveGreen}]{\#},
numbers=left,
numberstyle=\tiny,
% frame=tb,
columns=fullflexible,
showstringspaces=false,
tabsize=2
}
\lstdefinestyle{vhdlStyle}{language=vhdl}
\lstset{
language=vhdl,
%basicstyle=\small\ttfamily,
basicstyle=\small\ttfamily,
keywordstyle=\color{blue}\ttfamily,
stringstyle=\color{red}\ttfamily,
commentstyle=\color{magenta}\ttfamily,
morecomment=[l][\color{magenta}]{\#},
numbers=left,
numberstyle=\tiny,
% frame=tb,
columns=fullflexible,
showstringspaces=false,
tabsize=2
}
\lstdefinestyle{pythonStyle}{language=python}
\lstset{
language=python,
%basicstyle=\small\ttfamily,
basicstyle=\small\ttfamily,
keywordstyle=\color{blue}\ttfamily,
stringstyle=\color{red}\ttfamily,
commentstyle=\color{magenta}\ttfamily,
morecomment=[l][\color{magenta}]{\#},
numbers=left,
numberstyle=\tiny,
% frame=tb,
columns=fullflexible,
showstringspaces=false,
tabsize=2
}
% // -- source code listings --


\title{Bachelorprojekt}
\date{2018-11-21}
\author{Fabian Huber}

\begin{document}

% Titlepage for HAW lab report
\begin{titlepage}
\definecolor{blue(ncs)}{rgb}{0.0, 0.53, 0.74}
\begin{figure}[h!]
  \begin{flushright}
  \begin{spacing}{1.5}
  \includegraphics[width=.5\linewidth]{images/hawlogo.png}
  \label{fig:hawlogo}\\
  \small Fakultät Technik und Informatik\\
  \small Department Informations- und Elektrotechnik
  \end{spacing}
  \end{flushright}
\end{figure}
\textbf{\large Bachelorprojekt}
\begin{center}\noindent\textcolor{blue(ncs)}{\rule{13.5cm}{0.5mm}}\end{center}
\begin{spacing}{4.5}
\textbf{\huge Automated Driving}
\end{spacing}
\textbf{\large\indent RC Car Control with Open Source Image Processing}
\begin{center}\noindent\textcolor{blue(ncs)}{\rule{13.5cm}{0.5mm}}\end{center}
\begin{spacing}{1.15}
\vspace*{\fill}
\noindent
\textnormal{\\
  Prof. Dr.-Ing. Marc Hensel \\
  \textbf{Projektgruppe:} Fabian Huber, Enzo Morino, Markus Trockel \\
  \textbf{Abgabe:} 11.02.2019 \\
}
\end{spacing}
\end{titlepage}
% --- end of titlepage ---

  \pagenumbering{gobble}

	% --- Abstract/Kurzübersicht ---
	\newpage
	\section{Kurzübersicht}
Das Ziel des Projekts ist ein Modellauto, dass mit Kamera ,Abstandssensor und
Rapsberry Pi ausgestattet, durch Fahrbahnlinienerkennung teilautonom einem
Straßenverlauf foglen kann. Es ist eine pratkitsche Umsetzung von Systemarchitektur, 
Sensorik und der Nutzung von den Werkzeugen der Bildverabreitung der Librry openCV 
in Python.\\
Dafür ist ein Prozess entwickelt worden, bei dem aus Sensordaten
Steuerungsinformationen für den Wagen generiert werden. Die Durchführung hat
gezeigt, dass die Umsetzung möglich ist, dass für eine stabile Funktionsweise
weitere Entwicklungen von Funktionen nötig sind, die die zeitkritischen
Anforderungen erfüllen und hardwarebedingte Fehleranfälligkeiten abfangen, um so
eine rkontinuierliche Fahrt zu ermöglichen.\\

	
	% --- Inhaltsverzeichnis ---
  \newpage
  \tableofcontents
  \newpage

  \pagenumbering{arabic}



% --- Kapitel ---
	\newpage
  \section{Einleitung}
Im Jahr 2005 ging das Projektauto Stanley der Standford University an den Start
der DARPA Grand Challenge, einem Rennen autonomer Fahrezeuge über eine
definierte Strecke in der Mojave-Wüste. Nach 212,76km durch unwegsames Gelände
ging Stanley als Erster über die Ziellinie und das Team um Professor Sebastian
Thrun war Gewinner des mit 2 Millionen Dollar dotierten Rennens.\\
Die Entwicklung des autonomen Fahrens für Fahrzeuge des Straßenverkehrs ist ein
aktuelles Thema, das immer mehr in die Praxis umgesetzt wird. Dabei gibt es
verschiedenen Stufen des teilautonomen Fahrens. So gehört zur Stufe 2 die
Funktion, dass Lenkvorgänge vom Fahrassistenten ausgeführt werden.\\
Durch dieses Projekts soll ein Einblick in die Materie geschaffen
werden und die Umsetzbarkeit im Rahmen studentischer Möglichkeiten erfahren
werden.\\
Die Autoren wollen mit ihrem Auto Hawey einen Startschuß setzen und mit diesem
Bericht ein nachfolgendes Team inhaltlich in der Sache abholen, dass ein
Einstieg in das Projekt und so ein konstruktives Fortsetzten und Verbessern 
möglich ist.\\

Fabian Huber, Enzo Morino, Markus Trockel

 
 	\newpage
  \section{Ziel des Projekts}
Das Projekt hat als Ziel, ein Modellauto mit Elektroantrieb 
durch Fahrbahnlinienerkennung eine vorgegebene Strecke fahren zu lassen. Als
technische Bestandteile wird ein Raspberry Pi 3 mit dazugehöriger Kamera
gewählt, dazu ein Sensor zur Abstandsmessung, um etwaige Kollisionen zu
vermeiden. Implementiert wird der Code in Python unter der Nutzung der
Bildverarbeitungslibrary openbCV. \\
Das Auto soll  sowohl ein 4m lange Strecke gradeaus als auch in Kurven von 90° 
zwischen zwei Fahrbahnlinien die Spur halten. Idealerweise ist es möglich, dass 
der Wagen auch mit Linienunterbrechungen weitersteuern und so eine acht-förmige 
Strecke abfahren kann wie sie beim Carolo-Cup, einem Wettbewerb für autonome
Modellfahrzeuge, befahren wird.

  
  \section{Software}
    \ \\
    \subsection{Aufbau}
    \ \\
    \begin{minipage}{\columnwidth}
      \makeatletter
      \def\@captype{figure}
      \makeatother
      \centering
      \includegraphics[width=0.8\linewidth]{images/code-flowchart.png}
      \caption{Aufbau des Python Codes}
      \label{fig:image-01}
    \end{minipage}
    \ \\

    \subsection{Externe Module}
    \ \\
    \begin{minipage}{\columnwidth}
      \makeatletter
      \def\@captype{table}
      \makeatother
      \centering
      %\rowcolors{1}{grey}{white}
      \begin{tabular}{ l | l }
      % \multicolumn{2}{|c}{Frame \#} & \multicolumn{4}{|c}{LCD 0/3} &
      Name & Beschreibung \\ \hline \hline
      tkinter & ... \\
      Adafruit\_PCA9685 & Bibliothek zur Ansteuerung des Motorcontrollers \\
      numpy & Bibliothek zur Verwendung von Matlab Funktionen \\
      cv2 & OpenCV 2 bietet Algorithmen zur Bildverarbeitung \\
      io & ... \\
      time & ... \\
      importlib & ... \\
      argparse & ... \\
      pivideostream & ... \\
      picamera & ... \\
      threading & ... \\
      RPi.GPIO & Bibliothek zur Ansteuerung der GPIO ports des Raspbery Pi \\
      \end{tabular}
      \caption{verwendete externe Python Module}
      \label{tab:01}
    \end{minipage}
    
    \subsection{Eigene Module}
    \ \\
    \begin{minipage}{\columnwidth}
      \makeatletter
      \def\@captype{table}
      \makeatother
      \centering
      %\rowcolors{1}{grey}{white}
      \begin{tabular}{ l | l }
      % \multicolumn{2}{|c}{Frame \#} & \multicolumn{4}{|c}{LCD 0/3} &
      Name & Beschreibung \\ \hline \hline
      modAnalysis & Verantwortlich für die eigentliche Verarbeitung der visuellen Informationen \\
      modAct & Verantwortlich für die Ansteuerung des Motors und der Lenkung \\
      modCamera & Bereitet das Kamerabild für die Verarbeitung und Anzeige vor. \\
      modSonic & Kommuniziert mit dem Ultraschallsensor und liefert Distanz zum Hindernis.\\
      \end{tabular}
      \caption{verwendete eigene Python Module}
      \label{tab:01}
    \end{minipage}


  \newpage  
  \section{Prinzip der Steuerung} 	
  Über ein Python-Programm, welches auf dem Raspberry Pi läuft, wird der Video-Stream der angeschlossenen Kamera iterativ ausgewertet. Es werden die beiden Fahrbahnlinien erkannt, durch Geraden angenähert und deren Fluchtpunkt berechnet. Auf Grundlage der x-Koordinate dieses Fluchtpunktes wird ein Ausgangssignal berechnet, welches über das PWM-Modul den Servo ansteuert, und damit den Lenkwinkel festlegt.  


  %\section{Hardware}
  Für die technische Umsetzung werden eine Reihe von Komponenten verwendet.

  \subsection{Raspberry Pi 3}
  
    % image of the raspberry pi - source: http://www.raspberrypi.org
    \begin{minipage}{\columnwidth}
      \makeatletter
      \def\@captype{figure}
      \makeatother
      \centering
      \includegraphics[width=0.6\linewidth]{images/hw_raspberrypi3.jpg}
      \caption{Raspberry Pi 3 Model B}
      \label{fig:img-hw-01}
    \end{minipage}
    \vspace{0.5cm}
    Das Herzstück des Projekts bildet ein Raspberry Pi 3 Model B. Ausgestattet
    ist dieser Mini-PC neben den üblichen Schnittstellen mit einem Wifi Modul
    und verfügt

  \subsection{Motorcontroller}

    % image of the motor and servo controller module  - source: https://www.makerlab-electronics.com
    \begin{minipage}{\columnwidth}
      \makeatletter
      \def\@captype{figure}
      \makeatother
      \centering
      \includegraphics[width=0.6\linewidth]{images/hw_pca9685.jpg}
      \caption{PCA9685 Controller Modul}
      \label{fig:img-hw-02}
    \end{minipage}
    \vspace{0.5cm}

  \subsection{Ultraschallsensor}

    % image of ultrasonic sensor module  - source: https://www.makerfabs.com 
    \begin{figure}[H]
    \center
    \begin{subfigure}{.5\textwidth}
      \centering
      \includegraphics[width=.8\linewidth]{images/hw_hcsr04_01.png}
    \end{subfigure}%
    \begin{subfigure}{.5\textwidth}
      \centering
      \includegraphics[width=.8\linewidth]{images/hw_hcsr04_02.png}
    \end{subfigure}
      \caption{HC-SR04 Sensor Modul}
      \label{fig:img-hw-03}
    \end{figure}
    \vspace{0.5cm}

  \subsection{RC Fahrzeug}

    Ein RC Fahrzeug von Conrad ist auch verwendet worden, keine Ahnung, was das
    genau für eins war...
    \ \\

  
  \section{Software}
    \ \\
    \subsection{Aufbau}
    \ \\
    \begin{minipage}{\columnwidth}
      \makeatletter
      \def\@captype{figure}
      \makeatother
      \centering
      \includegraphics[width=0.8\linewidth]{images/code-flowchart.png}
      \caption{Aufbau des Python Codes}
      \label{fig:image-01}
    \end{minipage}
    \ \\

    \subsection{Externe Module}
    \ \\
    \begin{minipage}{\columnwidth}
      \makeatletter
      \def\@captype{table}
      \makeatother
      \centering
      %\rowcolors{1}{grey}{white}
      \begin{tabular}{ l | l }
      % \multicolumn{2}{|c}{Frame \#} & \multicolumn{4}{|c}{LCD 0/3} &
      Name & Beschreibung \\ \hline \hline
      tkinter & ... \\
      Adafruit\_PCA9685 & Bibliothek zur Ansteuerung des Motorcontrollers \\
      numpy & Bibliothek zur Verwendung von Matlab Funktionen \\
      cv2 & OpenCV 2 bietet Algorithmen zur Bildverarbeitung \\
      io & ... \\
      time & ... \\
      importlib & ... \\
      argparse & ... \\
      pivideostream & ... \\
      picamera & ... \\
      threading & ... \\
      RPi.GPIO & Bibliothek zur Ansteuerung der GPIO ports des Raspbery Pi \\
      \end{tabular}
      \caption{verwendete externe Python Module}
      \label{tab:01}
    \end{minipage}
    
    \subsection{Eigene Module}
    \ \\
    \begin{minipage}{\columnwidth}
      \makeatletter
      \def\@captype{table}
      \makeatother
      \centering
      %\rowcolors{1}{grey}{white}
      \begin{tabular}{ l | l }
      % \multicolumn{2}{|c}{Frame \#} & \multicolumn{4}{|c}{LCD 0/3} &
      Name & Beschreibung \\ \hline \hline
      modAnalysis & Verantwortlich für die eigentliche Verarbeitung der visuellen Informationen \\
      modAct & Verantwortlich für die Ansteuerung des Motors und der Lenkung \\
      modCamera & Bereitet das Kamerabild für die Verarbeitung und Anzeige vor. \\
      modSonic & Kommuniziert mit dem Ultraschallsensor und liefert Distanz zum Hindernis.\\
      \end{tabular}
      \caption{verwendete eigene Python Module}
      \label{tab:01}
    \end{minipage}


      \section{Werkzeuge der Bildverabreitung}
Die folgende Einleitung in die Thematik basiert auf der Lektüre des Buches
"Digitale Bildverarbeitung" von Burger, Burge \citep{burgeDigitBild}, das als 
Literatur für den Einstieg empfohlen wird. Eine zusammenfassende Wiedergabe für 
das Peojekt relevanter Themene ist im Folgenden nachzulesen.\\
Ein Bestandteil der Bildverarbeitung ist die Bildanalyse, bei der es darum geht,
sinnvolle Informationen aus Bildern zu extrahieren. Genauer ist der Bereich der
Computer Vision gefragt, die Sehvorgänge des Menschen in der dreidimensionalen Welt
zu mechanisieren.\\
Die Steuerung des Wagens soll durch Informationen aus den Kamerabildern
erfolgen: Durch Erkennen der Fahrbanhlinien soll die Lenkung innerhalb der Spur
geregelt werden.\\
Dieser Prozess lässt sich beschreiben mit:
\begin{enumerate}
  \item Digitalisierung mit Hilfe der Kamera
  \item Vorverarbeitung (Bildverbesserung bzw. Anpassung an den Zweck) durch
    Umwandlung in ein binäres Canny-Edges-Bild
  \item Segmentierung eine Vorauswahl des Bildauschnittes mit den relevante
    Informationen
  \item Merkmalsextraktion zur Linienerkennung durch die Hough-Transformation
  \item Parametrisierung als mathematische  Beschreibung der Fahrbahnlinien zur
    weiteren Informationsverarbeitung
\end{enumerate}

\subsection{Digitalisierung}
Der von der Kamera aufgenommen Bilder-Stream liegt in digitaler Form vor. Dabei lassen
sich benötigte Parameter eintellen(ergänzen Fabian?). Um ein optimales Bild zu
erhalten sind die Beleuchtungsumstände zu beachten, da hierdurch die
Differenzierung von Linien im Bild beeinflusst wird. Beachte timing? \\
In der Bildverabreitung ist der Nullpunkt der x- und y-Achse in der linken
oberen Ecke des Bildes definiert, was für ein eindeutiges Anwenden von Prozessen
und daraus generierten Informationen relevant ist.

\subsection{Vorverarbeitung- Canny-Kantenoperator}
Zur späteren Erkennung der Linien werden die dafür relevanten Informationen aus
dem Bild gefiltert: sogenannte Bildkannten. Bildkanten sind Übergangsstellen, wo
ein hoher Grauwertsprung von einem Pixel zum Nachbarpixeel erfolgt, wie z. B. bei 
einem weißen Farbahnstreifen zwei Kannten erkannt werden sollen.\\
Die Canny-Edges-Funktion aus der openCV Library ist ein bewährter Algorithmus zur Kantenerkennung, da drei Ziel
gleichzeitig erreicht werden drei Zile: ein zuverlässiges Detektieren
vorhandener Kanten, die Position der Kante präzise zu bestimmen, und Farbsprünge, 
die nicht als Kante interoretiert werden sollen, auszulassen.\\

\subsection{Segmentierung}
[BILD Kamera mit Fahrbahnlinien]
Der für die Erkennung der Fahrbahnlinien relevante Bereich liegt in einem
unteren Dreieck des Kamerabildes. Dieser Teil wird ausgewählt, bzw. davon ausserhalb
liegende Bereiche werden nicht bei der Erkennenung der Farbahnlinien
berücksichtigt.\\

\subsection{Merkmalsextraktion zur Linienerkennung}
Das Canny-Kantenbild wird mit Hilfe der Hough-Transformation aus der
openCV-Library in den Hough-Raum transformiert, wo dann die Erkennung der Linien
erfolgt.

\begin{minipage}{\columnwidth}
  \makeatletter
  \def\@captype{figure}
  \makeatother
  \centering
  \includegraphics[width=0.8\linewidth]{images/gradeThetaR.png}
  \caption{gradeThetaR.png}
  \label{fig:gradeThetaR}
\end{minipage}

Dabei kommt zur Anwendung, dass eine Gerade mathematisch sowohl mit $y
= m \dot x + b$ als auch mit $r(\theta) = x \cdot cos(\theta) + y \cdot
sin(\theta)$ beschrieben werden kann. Dabei ist in der zweiten Beschreibung 
$\theta$ der  Winkel des Radius $r$ zum Urspung, an dessen Ende senkrecht die
beschriebene Grade verläuft. Zu Bedenken ist, dass wie schon erwähnt der
Koordinatenursprung des Bildes oben links definiert ist.\\
Das rechenaufwendige Verfahren der Hough-Transformation generiert für jeden
Kantenbildpunkt des Canny-Edges-Bildes Bildpunkte im Hough-Raum, dessen Achsen
ein $\theta$ / $r$ - Koordinatensystem bilden. Linien im kartesisches System
sind als Punkthäufungen im Hough-Raum erkennbar. Liegt eine Anzahl von
Punkten oberhalb eines zu definierenden Schwellwertes, wird eine Linie erkannt
und die Houh-Transformationsfunktion gibt ein Wertpaar $\theta$, $r$ zurück.




  \subsection{OpenCV}  
\glqq OpenCV ist eine freie Programmbibliothek mit Algorithmen für die
Bildverarbeitung und maschinelles Sehen. Sie ist für die Programmiersprachen C,
C++, Python und Java geschrieben und steht als freie Software unter den
Bedingungen der BSD-Lizenz. \grqq

\subsubsection{Canny Edges Filter}
Die Funktion wird im Kamera-Modul eingesetzt, um aus dem Kamera-Stream eine
Stream nur mit Umrissen zu konvertieren. Dafür wird zwischen benachbarten Pixel
die Grauwert-Differenz als Steigung, im 2-dimensinalen der sogeannte Gradient,
bestimmt. Auf Grund der Intensität des Gradienten kann eine Stelle als Kante
klassifiert werden.\\
Es  werden folgende Argumente genutzt:\\
canny = cv.Canny(image, minVal, maxVal)

[cannyThresholds.png]

Die Argumente Nr. 2 und 3 setzen einen unteren und oberen Threshold. Der untere
Wert beschreibt eine Grenze für Gradienten, die kleiner sind und nicht als
Kanten erkannt werden sollen. Gradienten, die oberhalb des maxVal liegen, werden
als Kanten erkannt. Gradienten, die zwischen den beiden Werten liegen, werden,
wenn sie mit Gradienten ausserhalb der Grenzen verbunden sind, denen zugeordnet,
und können sowohl im Bild erscheinen als auch nicht. Sind sie nicht verbunden,
werden sie nicht dargestellt.
Für die Anwendung haben sich die Grenzen von 80 und 140 als passend herausgestellt.

\subsubsection{Hough Space Transformation}
... hier weiter
  
  
  \section{Werkzeuge der Bildverabreitung}
Die folgende Zusammenfassung der Thematik basiert auf dem Buch
"Digitale Bildverarbeitung" von Burger, Burge \citep{BurgeDigitBild}, das als 
Standartliteratur zu empfehlen ist. Eine zusammenfassende Wiedergabe der für 
das Projekt relevanter Themen ist im Folgenden nachzulesen.\\
Ein Bestandteil der Bildverarbeitung ist die Bildanalyse, bei der
sinnvolle Informationen aus Bildern extrahiert werden. Genauer ist der Themenbereich
\textit{Computer Vision} gemeint, bei dem es um das Mechanisieren von
Sehvorgängen des Menschen in der dreidimensionalen Welt geht.\\
Die Steuerung des Wagens soll durch Informationen aus den Kamerabildern
erfolgen: Durch Erkennen der Fahrbahnlinien wird der Lenkungsservo geregelt, der
das Auto innerhalb der Spur zwischen den Fahrpahnlinien hält.\\
Dieser Prozess lässt sich in Einzelschritten beschreiben mit:
\begin{enumerate}
  \item \textit{Digitalisierung} der dreidimensionalen Welt mit Hilfe der Kamera
  \item \textit{Vorverarbeitung} (Bildverbesserung bzw. Anpassung an den Zweck) durch
    Umwandlung in ein binäres Canny-Edges-Bild
  \item \textit{Segmentierung} durch Vorauswahl des Bildauschnittes mit den relevante
    Informationen
  \item \textit{Merkmalsextraktion} zur Linienerkennung durch die Hough-Transformation
  \item \textit{Parametrisierung} als mathematische  Beschreibung der Fahrbahnlinien zur
    weiteren Informationsverarbeitung
\end{enumerate}

\subsection{Digitalisierung}
Der von der Kamera aufgenommen Bilder-Stream liegt in digitaler Form vor. Dabei lassen
sich benötigte Parameter eintellen. Um ein optimales Bild zu
erhalten sind die Beleuchtungsumstände zu beachten, da hierdurch die
Differenzierung von Linien im Bild erheblich beeinflusst wird.\\
Anzumerken ist hier, dass in der Bildverabreitung der Ursprung des
Koordinatensystems bzw. der x- und y-Achse in der linken
oberen Ecke des Bildes definiert ist.

\subsection{Vorverarbeitung - Der Canny-Kantenoperator}
Zur späteren Erkennung der Linien werden die dafür relevanten Informationen aus
dem Bild gefiltert: sogenannte Bildkannten. Bildkanten sind Übergangsstellen, wo
ein hoher Grauwertsprung von einem Pixel zum Nachbarpixel vorliegt, wie z. B. bei 
einem weißen Farbahnstreifen auf dunkler Fahrbahn, hier werden zwei Kannten
analysiert. Kanten werden dem entsprechend an diversen Stellen im Bild
detektiert, deshalb sind die Parameter entsprechend zu wählen.\\
Die Canny-Edges-Funktion ist ein bewährter Algorithmus zur Kantenerkennung, da drei Ziel
gleichzeitig erreicht werden: ein zuverlässiges Detektieren
vorhandener Kanten, die Position der Kante präzise zu bestimmen, und Farbsprünge, 
die nicht als Kante interpretiert werden sollen, auszulassen. Diese Funktion
ist in der OpenCV-Library enthalten. \\

\subsection{Segmentierung}
[BILD Kamera mit Fahrbahnlinien]
Der für die Erkennung der Fahrbahnlinien relevante Bereich liegt in einem
unteren Dreieck des Kamerabildes. Dieser Teil wird ausgewählt, bzw. davon ausserhalb
liegende Bereiche werden nicht berücksichtigt.\\

\subsection{Merkmalsextraktion zur Linienerkennung}
Das Canny-Kantenbild wird mit Hilfe der Hough-Transformation aus der
openCV-Library in den Hough-Raum transformiert, wo dann die Erkennung der Linien
erfolgt.

\begin{minipage}{\columnwidth}
  \makeatletter
  \def\@captype{figure}
  \makeatother
  \centering
  \includegraphics[width=0.8\linewidth]{images/gradeThetaR.png}
  \caption{Beschreibung einer Graden durch Winkel und Abstand vom Ursprung}
  \label{fig:gradeThetaR}
\end{minipage}

\vspace{0.8cm}

Eine Gerade kann  mathematisch sowohl mit $y = m \cdot x + b$ als auch mit 
$r(\theta) = x \cdot cos(\theta) + y \cdot sin(\theta)$ beschrieben werden. 
Dabei ist in der zweiten Schreibweise $\theta$ der  Winkel des Radius $r$ zum 
Urspung, an dessen Ende senkrecht die beschriebene Grade verläuft. Zu Bedenken 
ist, dass, wie schon erwähnt, der Koordinatenursprung des Bildes oben links 
definiert ist.\\
Das rechenaufwendige Verfahren der Hough-Transformation generiert für jeden
Kantenbildpunkt des Canny-Edges-Bildes Bildpunkte im Hough-Raum, dessen Achsen
ein $\theta$ / $r$ - Koordinatensystem bilden - eine Gerade wird also durch
einen Punkt dargestellt.  Teilstücke einer Geraden im kartesisches System
sind als Punkthäufungen im Hough-Raum erkennbar. Liegt eine Anzahl von
Punkten oberhalb eines zu definierenden Schwellwertes, wird eine Linie als 
erkannt gewertet, und die Hough-Transformationsfunktion gibt ein Wertpaar 
$\theta$, $r$ zurück.
 
  
  \section{Iterative Auswertung des Kamerabildes}	

	Das Bild der Kamera werden iterativ ausgewertet um die beiden Fahrbahnlinien zu erkennen und daraus ein Ausgangssignal für die Steuerung zu generieren. Der gewählte Ansatz war es, die Fahrbahnlinien durch zwei Geraden anzunähern. Die Funktion HoughLines wird verwendet um die Fahrbahnlinien zu erkennen und die zugehörigen Geraden zu berechnen. Da die Funktion HoughLines in den meisten Fällen nicht nur die beiden Fahrbahnlinien erkennt, sondern alle möglichen Geraden, bestand die erste Aufgabe darin, nur die Geraden herauszufiltern, welche die Fahrbahnlinien repräsentieren, und alle anderen Geraden zu verwerfen. Im Programm wird dies erreicht, indem alle Geraden, die durch die Funktion HoughLines gefunden wurden mit dem Geradenpaar aus dem vorherigen Programmdurchlauf verglichen werden, und nur ähnliche Geraden beibehalten werden.\\
	
	Beim Start des Programm werden für die beiden Geraden feste Werte vorgegeben, die dann korrigiert und an die erkannten Linien angepasst werden. In Abbildung \ref{fig:fahrbahn} ist dies bildlich dargestellt. Die roten Linien sind zu Beginn des Programms fest vorgegeben. Die blauen Linien sind das Ergebnis der Korrektur durch das Programm.
	
	\begin{figure}[H]
		\centering
		\includegraphics[width=.5\linewidth]{images/fahrbahn.png}
		\caption{Beispiel für die erste Form der Geradenrepräsentation}
		\label{fig:fahrbahn}
	\end{figure}
	
	
	Der Vergleich der Geraden wird durchgeführt, indem um die beiden Geraden aus dem letzten Durchlauf ein Toleranz-Fenster gelegt wird, und von allen gefundenen Geraden nur diejenigen herausgefiltert werden, die innerhalb dieses Toleranz-Fensters liegen. Von diesen Geraden wird dann der Durschnitt berechnet, sodass nur noch zwei Geraden für die beiden Fahrbahnlinien übrig bleiben. \\
	
	Eine Schwierigkeit besteht darin die Geraden so zu repräsentieren, dass diese gut miteinander vergleichbar sind. Bei einer geringen Änderung einer Geraden von einem zum nächsten Zyklus sollen sich deren Parameter dabei auch nur geringfügig ändern. Ansonsten werden ähnliche Geraden beim filtern verworfen, da deren Parameter nicht im Toleranzfenster liegen. Im Programm wurden daher je nach Anwendungsfall verschiedene Repräsentationsformen verwendet, welche im folgenden beschrieben werden.
	
	\subsection{Verwendete Repräsentationsformen von Geraden}
	
	Die Funktion HoughLines gibt als Ergebnis eine Liste von Geraden aus, die durch die beiden Parameter $\rho$ und $\theta$ beschrieben werden wird. $\theta$ ist der Winkel zwischen der Normalen, die senkrecht auf der Geraden steht, und der x-Achse. Der Winkel  $\rho$ ist die Länge dieser Normalen, d.h. der Abstand zwischen der Geraden und dem Ursprung des Koordinatensystems. Der Winkel 
	Abbildung \ref{fig:rho_theta1} zeigt ein Beispiel für die Repräsentation zweier Fahrbahnlinien in der $\rho$-$\theta$-Darstellung, für den Fall, dass der Wagen relativ mittig und gerade auf der Fahrbahn steht. Das Rechteck symbolisiert die Ränder des Kamerabildes. Der Koordinatenursprung liegt in der oberen linken Ecke. $\theta1$ ist positiv, $\theta2$ negativ. $\rho1$ und $\rho2$ sind jeweils die Abstände der Geraden vom Koordinatenursprung.
	
	\begin{figure}[H]
		\centering
		\includegraphics[width=.5\linewidth]{images/rho_theta1.jpg}
		\caption{Beispiel für die erste Form der Geradenrepräsentation}
		\label{fig:rho_theta1}
	\end{figure}
	
	Eine Schwäche dieser Repräsentationsform wird liegt darin, dass zwei sehr ähnliche Geraden komplett verschiedene Winkel $\theta$ haben können. Dies wird in Abbildung \ref{fig:rho_theta2} deutlich. $\theta1$ beträgt hier ca. $70^\circ$, $\theta2$ ca. $-100^\circ$.
	
	
	
	\begin{figure}[H]
		\centering
		\includegraphics[width=.5\linewidth]{images/rho_theta2.jpg}
		\caption{Problematischer Fall für diese Form der Geradenrepräsentation}
		\label{fig:rho_theta2}
	\end{figure}

	Um dieses Problem zu umgehen wurden die Geraden mithilfe der Funktion get\_offset\_alpha1 in eine andere Form umgerechnet, bei der diese durch den Offset $\omega$ auf der y-Achse und den Winkel zur x-Achse beschrieben werden. Die Umrechnung von der $\rho$-$\theta$-Form erfolgt folgendermaßen:\\

	
	
	Für Winkel $\theta<90^\circ$ (linke Fahrbahnlinie):
	
	\begin{align*}
	\alpha=90^{\circ}-\theta
	\omega&=-\frac{\rho}{\cos{90^{\circ}-\theta}} \\
	\end{align*}
	
	Für Winkel $\theta>90^\circ$ (rechte Fahrbahnlinie):
	
	\begin{align*}
	\alpha=\theta-90^{\circ}
	\omega&=-\frac{\rho}{\cos{\theta-90^{\circ}}} \\
	\end{align*}
	
	Abbildung \ref{fig:alpha_omega1} veranschaulicht die Repräsentation der beiden Fahrbahnlinien in dieser Form.
	
	\begin{figure}[H]
		\centering
		\includegraphics[width=.5\linewidth]{images/alpha_omega1.jpg}
		\caption{Beispiel für die erste Form der Geradenrepräsentation}
		\label{fig:alpha_omega1}
	\end{figure}

	Auch bei dieser Form ergibt sich das Problem, dass sich bei geringfügiger Veränderung der Geraden deren Parameter in bestimmten Situationen stark ändern. Dies ist besonders für die rechte Fahrbahnlinie der Fall, wie sich in Abbildung \ref{fig:alpha_omega2} erkennen lässt. Hier sieht man, dass sich der Offset $\omega$ stark ändert, wenn sich der Winkel der Geraden leicht ändert.


	\begin{figure}[H]
		\centering
		\includegraphics[width=.3\linewidth]{images/alpha_omega2.jpg}
		\caption{Beispiel für die erste Form der Geradenrepräsentation}
		\label{fig:alpha_omega2}
	\end{figure}

	Daher wurden für die beiden Gerade zwei verschiedene Repräsentationsformen gewählt. Für die Linke gerade wird der Offset von der linken oberen Ecke gemessen, für die rechte der Offset von der rechten oberen Ecke. Die Winkel werden wie gehabt zur x-Achse gemessen. In dieser Form lassen sich die Geraden am besten vergleichen. Die Umformung von der $\rho$-$\theta$-Form geschieht in der Funktion get\_offset\_alpha2 durch folgende Berechnungen: \\
	
	\begin{align*}
	\alpha=90^{\circ}-\theta
	\end{align*}
	
	
	Für Winkel $\theta<90^\circ$ (linke Fahrbahnlinie):
	
	\begin{align*}
	\omega&=-\frac{\rho}{\cos{90^{\circ}-\theta}} \\
	\end{align*}
	
	Für Winkel $\theta>90^\circ$ (rechte Fahrbahnlinie):
	
	\begin{align*}
	\omega&=\frac{B+\frac{\rho}{\cos{\theta-90^{\circ}}}}{\tan{90^{\circ}-\theta}} \\
	\end{align*}
	
	Mit der Bildbreite B.
	
	Abbildung \ref{fig:alpha_omega3} veranschaulicht diese Repräsentationsform.
	
		\begin{figure}[H]
		\centering
		\includegraphics[width=.5\linewidth]{images/alpha_omega3.jpg}
		\caption{Beispiel für die erste Form der Geradenrepräsentation}
		\label{fig:alpha_omega3}
	\end{figure}


	\subsection{Berechnung des Schnittpunktes der beiden Geraden}
	
	Der Lenkwinkel wird mithilfe der x-Koordinate des Schnittpunktes der beiden Geraden berechnet. Die Abweichung dieses x-Wertes vom Mittelpunkt kann als Regeldifferenz verwendet werden.\\
	Die x-Koordinate des Schnittpunktes (engl. vanishing point) wird mit folgender Formel berechnet:\\
	
	\begin{align*}
	vp_x=\frac{\omega_R-\omega_L}{\tan{\alpha_L}+\tan{\alpha_R}} \\
	\end{align*}
	
	Um die Abweichung vom x-Wert der Bildmitte zu erhalten, wird von diesem Wert die Hälfte der Bildbreite B abgezogen:\\
	
	\begin{align*}
	Dx=vp_x-\frac{B}{2} \\
	\end{align*}
	
	
	
	
	
	
	
	
%	Beim Start des Programms werden für die beiden Geraden feste Werte vorgegeben. Anhand dieser Anfangswerte werden die im nächsten Durchlauf die Fahrbahnlinien durch bessere Werte angenähert.\\
%	Dafür wird das Bild zunächst mithilfe eines Schwellwertes in Schwarzweiß umgewandelt. Bei einem geeigneten Schwellwert sollten nun möglichst nur die beiden Fahrbahnlinien weiß sein (den Wert 1 haben).\\
%	Als nächstes wird die Funktion HoughLines aus der Bibliothek cv ausgeführt, die alle Geraden im Bild sucht. Die Funktion Houghlines gibt als Rückgabewert eine Liste von Geraden zurück, die in Polarkoordinaten durch einen Winkel $\rho$ und ein Abstand $\theta$ repräsentiert werden. $\rho$ ist der Winkel zwischen der Normalen, die senkrecht auf der Geraden steht, und der x-Achse. $\theta$ ist die Länge dieser Normalen, d.h. der Abstand zwischen der Geraden und dem Ursprung des Koordinatensystems.
	
  
  \section{Ansteuerung von Servo und Motorcontroller}


Motorcontroller und Servo werden mithilfe eines Adafruit PCA9685 16-Kanal Servo Treibers angesteuert. Dieser nimmt Befehle des Raspberry Pi's über $I^2C$ entgegen und wandelt sie in ein PWM-Signal um. Das PWM Signal hat eine Periodendauer von 20ms, die Einschaltdauer $T_{on}$ liegt zwischen 1ms und 2ms. \\
Der Servotreiber hat eine eigene Bibliothek aus der am wichtigsten die Funktion set\_pwm(kanal, on,off) ist. Als Kanal wurde für den Servo 0 gewählt, für den Motorcontroller 1. Der Einschaltzeitpunt ''on'' wird immer auf 0 gesetzt. Für den Auschaltzeitpunkt kann theoretisch eine Zahl zwischen 0 und 4095 gesetzt werden. Da beim Servo das On-Signal zwischen einer und zwei Millisekunden lang sein muss ergibt sich der theoretische Wert folgendermaßen: \\

\begin{align*}
	off&=T_{on}\cdot\frac{20ms}{4096}\\
	off_{min}&=1ms\cdot\frac{20ms}{4096} =204,8\\
	off_{max}&=2ms\cdot\frac{20ms}{4096} = 409,6
\end{align*}

Für das verwendete Auto wurde der Min- und Maxwert empirisch ermittelt indem die geschaut wurde, bei welchen Werten die Räder bis zum Anschlag ausgelenkt sind. Es wurden folgende Werte ermittelt:\\
Vollausschlag rechts: 272\\
Vollausschlag rechts: 342\\
Mittelposition: 307\\

Für den Vollausschlag nach rechts und links wurden die Werte 297 und 317 Verwendet, damit nicht zu scharfe Kurven gefahren werden.\\

Ebenso wurden für den Motor die Werte für den off-Parameter durch Ausprobieren ermittelt, indem geschaut wurde, bei welchen Werten das Auto mit angemessener Geschwindigkeit fährt. Hier wurden ermittelt:\\
Vorwärts: 320\\
Rückwärts: 295\\
Stop: 307\\


	
	
	
	
	
	
%	Beim Start des Programms werden für die beiden Geraden feste Werte vorgegeben. Anhand dieser Anfangswerte werden die im nächsten Durchlauf die Fahrbahnlinien durch bessere Werte angenähert.\\
%	Dafür wird das Bild zunächst mithilfe eines Schwellwertes in Schwarzweiß umgewandelt. Bei einem geeigneten Schwellwert sollten nun möglichst nur die beiden Fahrbahnlinien weiß sein (den Wert 1 haben).\\
%	Als nächstes wird die Funktion HoughLines aus der Bibliothek cv ausgeführt, die alle Geraden im Bild sucht. Die Funktion Houghlines gibt als Rückgabewert eine Liste von Geraden zurück, die in Polarkoordinaten durch einen Winkel $\rho$ und ein Abstand $\theta$ repräsentiert werden. $\rho$ ist der Winkel zwischen der Normalen, die senkrecht auf der Geraden steht, und der x-Achse. $\theta$ ist die Länge dieser Normalen, d.h. der Abstand zwischen der Geraden und dem Ursprung des Koordinatensystems.
	
 
	
	

  % \bibliography{hawey-documentation}
  % \bibliographystyle{ieeetr}

  % Bibliograpy / Literaturverzeichnis
  \newpage
  %\bibliography{literatur}
  \addcontentsline{toc}{section}{Literaturverzeichnis}
  % display bibliography
 
\end{document}
