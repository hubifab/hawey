\documentclass[a4paper,12pt]{article}

% include headers and preamble for reoport
% file: includes-report.tex
% -----------------------------------------------------------------------------
% includes for studies report
% -----------------------------------------------------------------------------

\usepackage{amsmath}
\usepackage{setspace}
\usepackage[top=1.2in, bottom=1.2in]{geometry}
\usepackage[x11names]{xcolor}

\usepackage{graphicx}
\usepackage[utf8]{inputenc}
\usepackage{siunitx}

%\usepackage{multirow}
%\usepackage{pgfplots}
\usepackage{subcaption}

% // -- for source code listings --
\usepackage{color}
\usepackage{xcolor}
\definecolor{OliveGreen}{RGB}{0,128,0}
\usepackage{listings}
\usepackage{caption}
\DeclareCaptionFont{white}{\color{white}}
\DeclareCaptionFormat{listing}{\colorbox{gray}{\parbox{\textwidth}{#1#2#3}}}
\captionsetup[lstlisting]{format=listing,labelfont=white,textfont=white}


\lstdefinestyle{cStyle}{language=C}
\lstset{
language=C,
%basicstyle=\small\ttfamily,
basicstyle=\small\ttfamily,
keywordstyle=\color{blue}\ttfamily,
stringstyle=\color{red}\ttfamily,
commentstyle=\color{magenta}\ttfamily,
morecomment=[l][\color{magenta}]{\#},
numbers=left,
numberstyle=\tiny,
% frame=tb,
columns=fullflexible,
showstringspaces=false,
tabsize=2
}
\usepackage{matlab-prettifier}
\lstdefinestyle{matlabStyle}{language=matlab}
\lstset{
%style=Matlab-editor,
language=matlab,
%basicstyle=\small\ttfamily,
basicstyle=\small\ttfamily,
keywordstyle=\color{blue}\ttfamily,
stringstyle=\color{red}\ttfamily,
commentstyle=\color{OliveGreen}\ttfamily,
morecomment=[l][\color{OliveGreen}]{\#},
numbers=left,
numberstyle=\tiny,
% frame=tb,
columns=fullflexible,
showstringspaces=false,
tabsize=2
}
\lstdefinestyle{vhdlStyle}{language=vhdl}
\lstset{
language=vhdl,
%basicstyle=\small\ttfamily,
basicstyle=\small\ttfamily,
keywordstyle=\color{blue}\ttfamily,
stringstyle=\color{red}\ttfamily,
commentstyle=\color{magenta}\ttfamily,
morecomment=[l][\color{magenta}]{\#},
numbers=left,
numberstyle=\tiny,
% frame=tb,
columns=fullflexible,
showstringspaces=false,
tabsize=2
}
\lstdefinestyle{pythonStyle}{language=python}
\lstset{
language=python,
%basicstyle=\small\ttfamily,
basicstyle=\small\ttfamily,
keywordstyle=\color{blue}\ttfamily,
stringstyle=\color{red}\ttfamily,
commentstyle=\color{magenta}\ttfamily,
morecomment=[l][\color{magenta}]{\#},
numbers=left,
numberstyle=\tiny,
% frame=tb,
columns=fullflexible,
showstringspaces=false,
tabsize=2
}
% // -- source code listings --


\title{Bachelorprojekt}
\date{2018-11-21}
\author{Fabian Huber}

\begin{document}

% Titlepage for HAW lab report
\begin{titlepage}
\definecolor{blue(ncs)}{rgb}{0.0, 0.53, 0.74}
\begin{figure}[h!]
  \begin{flushright}
  \begin{spacing}{1.5}
  \includegraphics[width=.5\linewidth]{images/hawlogo.png}
  \label{fig:hawlogo}\\
  \small Fakultät Technik und Informatik\\
  \small Department Informations- und Elektrotechnik
  \end{spacing}
  \end{flushright}
\end{figure}
\textbf{\large Bachelorprojekt}
\begin{center}\noindent\textcolor{blue(ncs)}{\rule{13.5cm}{0.5mm}}\end{center}
\begin{spacing}{4.5}
\textbf{\huge Automated Driving}
\end{spacing}
\textbf{\large\indent RC Car Control with Open Source Image Processing}
\begin{center}\noindent\textcolor{blue(ncs)}{\rule{13.5cm}{0.5mm}}\end{center}
\begin{spacing}{1.15}
\vspace*{\fill}
\noindent
\textnormal{\\
  Prof. Dr.-Ing. Marc Hensel \\
  \textbf{Projektgruppe:} Fabian Huber, Enzo Morino, Markus Trockel \\
  \textbf{Abgabe:} DD.MM.2019 \\
}
\end{spacing}
\end{titlepage}
% --- end of titlepage ---

  \pagenumbering{gobble}
  \newpage

  \tableofcontents
  \newpage

  \pagenumbering{arabic}

  \section{Einleitung}
    \ \\

  \section{Ziel des Projekts}
    \ \\
    
    Ziel des Projekts ist es, ein Modell-Auto zu bauen bzw. zu programmieren, das mithilfe eines Raspberry Pi's und einer Kamera autonom zwischen zwei Fahrbahnlinien die Spur halten kann. Das Auto soll auf einer 4 Meter langen geraden Strecke auf der Fahrbahn bleiben, einer Rechts- und einer Linkskurve mit angemessen großem Radius um 90° folgen, und eine Kreisfahrt in beide Richtungen beherrschen können. Optional soll auch eine 8-förmige Strecke wie beim Carolo-Cup befahren werden.
  
  \section{Kurzübersicht}
    \ \\
    
    
  \section{Prinzip der Steuerung}
  	\ \\
  	
  Über ein Python-Programm, welches auf dem Raspberry Pi läuft, wird der Video-Stream der angeschlossenen Kamera iterativ ausgewertet. Es werden die beiden Fahrbahnlinien erkannt, durch Geraden angenähert und deren Fluchtpunkt berechnet. Auf Grundlage der x-Koordinate dieses Fluchtpunktes wird ein Ausgangssignal berechnet, welches über das PWM-Modul den Servo ansteuert, und damit den Lenkwinkel festlegt.  
  	

  \section{Hardware}
    \ \\
    \subsection{Raspberry Pi 3}
    \ \\
    \subsection{Motorcontroller}
    \ \\
    \subsection{Ultraschallsensor}
    \ \\
    \subsection{RC Fahrzeug}
    \ \\
  
  \section{Software}
    \ \\
    \subsection{Aufbau}
    \ \\
    \begin{minipage}{\columnwidth}
      \makeatletter
      \def\@captype{figure}
      \makeatother
      \centering
      \includegraphics[width=0.8\linewidth]{images/code-flowchart.png}
      \caption{Aufbau des Python Codes}
      \label{fig:image-01}
    \end{minipage}
    \ \\

    \subsection{Externe Module}
    \ \\
    \begin{minipage}{\columnwidth}
      \makeatletter
      \def\@captype{table}
      \makeatother
      \centering
      %\rowcolors{1}{grey}{white}
      \begin{tabular}{ l | l }
      % \multicolumn{2}{|c}{Frame \#} & \multicolumn{4}{|c}{LCD 0/3} &
      Name & Beschreibung \\ \hline \hline
      tkinter & ... \\
      Adafruit\_PCA9685 & Bibliothek zur Ansteuerung des Motorcontrollers \\
      numpy & Bibliothek zur Verwendung von Matlab Funktionen \\
      cv2 & OpenCV 2 bietet Algorithmen zur Bildverarbeitung \\
      io & ... \\
      time & ... \\
      importlib & ... \\
      argparse & ... \\
      pivideostream & ... \\
      picamera & ... \\
      threading & ... \\
      RPi.GPIO & Bibliothek zur Ansteuerung der GPIO ports des Raspbery Pi \\
      \end{tabular}
      \caption{verwendete externe Python Module}
      \label{tab:01}
    \end{minipage}
    
    \subsection{Eigene Module}
    \ \\
    \begin{minipage}{\columnwidth}
      \makeatletter
      \def\@captype{table}
      \makeatother
      \centering
      %\rowcolors{1}{grey}{white}
      \begin{tabular}{ l | l }
      % \multicolumn{2}{|c}{Frame \#} & \multicolumn{4}{|c}{LCD 0/3} &
      Name & Beschreibung \\ \hline \hline
      modAnalysis & Verantwortlich für die eigentliche Verarbeitung der visuellen Informationen \\
      modAct & Verantwortlich für die Ansteuerung des Motors und der Lenkung \\
      modCamera & Bereitet das Kamerabild für die Verarbeitung und Anzeige vor. \\
      modSonic & Kommuniziert mit dem Ultraschallsensor und liefert Distanz zum Hindernis.\\
      \end{tabular}
      \caption{verwendete eigene Python Module}
      \label{tab:01}
    \end{minipage}




	\subsection{Auswertung des Kamerabildes}	

	Das Bild der Kamera wird iterativ ausgewertet um die beiden Fahrbahnlinien zu erkennen und daraus ein Ausgangssignal für die Steuerung zu generieren. Die Fahrbahnlinien werden durch zwei Geraden angenähert. Die beiden Geraden werden in jedem Durchlauf des Programms berechnet, indem die alle Geraden, die durch die Funktion HoughLines gefunden wurden mit dem Geradenpaar aus dem vorherigen Durchlauf verglichen werden. Durch diese vorgehensweise soll erreicht werden, dass nur die beiden Fahrbahnlinien erkannt werden und keine anderen Linien im Kamerabild. \\
	Der Vergleich der Geraden wird durchgeführt, indem um die Geraden-Parameter aus dem letzten Durchlauf ein Toleranz-Fenster gelegt wird, und für die neuen gefundenen Geraden geschaut wird, ob sie innerhalb dieses Toleranz-Fensters liegen.
	Eine Schwierigkeit besteht darin die Geraden so zu repräsentieren, dass diese gut miteinander vergleichbar sind. Es kann nämlich leicht passieren, dass bei einer bestimmten Repräsentation zwei sehr ähnliche Geraden komplett verschiedene Parameter haben, und daher nicht als ähnlich erkannt werden. Daher wurden im Programm verschiedene Repräsentationsformen verwendet, welche im folgenden beschrieben werden.
	
	\subsection{Verwendete Repräsentationsformen von Geraden}
	
	Form 1: Diese wird von der Funktion Houghlines verwendet. Sie besteht aus einem Winkel $\rho$ und einem Abstand $\theta$. $\theta$ ist der Winkel zwischen der Normalen, die senkrecht auf der Geraden steht, und der x-Achse. Der Winkel  $\rho$ ist die Länge dieser Normalen, d.h. der Abstand zwischen der Geraden und dem Ursprung des Koordinatensystems.
	Abbildung \ref{fig:rho_theta1} zeigt ein Beispiel für die Repräsentation zweier Fahrbahnlinien mittel zweier Winkel $\theta1$, $\theta2$, $\rho1$ und $\rho2$.
	
	\begin{figure}[h!]
		\centering
		\includegraphics[width=.5\linewidth]{images/rho_theta1.jpg}
		\label{fig:rho_theta1}
		\caption{Beispiel für die erste Form der Geradenrepräsentation}
	\end{figure}
	
	
	
	
	
	Beim Start des Programms werden für die beiden Geraden feste Werte vorgegeben. Anhand dieser Anfangswerte werden die im nächsten Durchlauf die Fahrbahnlinien durch bessere Werte angenähert.\\
	Dafür wird das Bild zunächst mithilfe eines Schwellwertes in Schwarzweiß umgewandelt. Bei einem geeigneten Schwellwert sollten nun möglichst nur die beiden Fahrbahnlinien weiß sein (den Wert 1 haben).\\
	Als nächstes wird die Funktion HoughLines aus der Bibliothek cv ausgeführt, die alle Geraden im Bild sucht. Die Funktion Houghlines gibt als Rückgabewert eine Liste von Geraden zurück, die in Polarkoordinaten durch einen Winkel $\rho$ und ein Abstand $\theta$ repräsentiert werden. $\rho$ ist der Winkel zwischen der Normalen, die senkrecht auf der Geraden steht, und der x-Achse. $\theta$ ist die Länge dieser Normalen, d.h. der Abstand zwischen der Geraden und dem Ursprung des Koordinatensystems.
	
 
	
	


  % \bibliography{hawey-documentation}
  % \bibliographystyle{ieeetr}

\end{document}
