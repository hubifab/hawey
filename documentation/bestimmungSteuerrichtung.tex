\subsection{Bestimmung der Steuerrichtung aus dem Fahrbahnlinienverlauf}
Das Kamerabild beinhaltet neben den relevanten Fahrbahnlininen weitere
Strukturen von z. B. Stühlen und Tischen im Raum, die als Linien erkannt werden.
Um die Auswahl auf die Fahrbahnlinien zu beschränken ist folgende Funktionalität
entsickelt worden:
\subsubsection{Initialisierung}
Wird der Wagen gestartet wird ein erstes Initialisierungsbild aufgenommen und
die darin detektierten Linien ausgewählt die in einem definierten
Toleranzbereich liegen. Aus diesen Linien wird für beide Fahrbahnseiten eine
Mittelwertlinie bestimmt. Der gewählte Toleranzbereich ist größer gewählt als
der des kontinuierlichen Bereichs, damit überhaupt Linien erkannt werden und ein
Prozeß gestartet werden kann.\\
\subsubsection{Kontinuolierlicher Fahrverlauf}
Für den Fahrtmodus ist dieser Toleranzbereich kleiner gewählt worden. Es wird in
einem iterativen Prozess bei jedem neuen Aufruf die aktuell detektierten Linien
in einem kleinen Tolranzbereich ausgewählt, der im Bereich der im vorangegangen
Schritt erkannt worden ist.\\
Aus den dabei neu entdeckten Linien wird auch hier der Mittelwert gebildet und
als neue Linie zur Bestimmung der Fahrrichtung jeweils eine linke und rechte
Hauptlinie bestimmt.\\
