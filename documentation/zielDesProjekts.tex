\section{Ziel des Projekts}
Das Projekt hat als Ziel, ein Modellauto mit Elektroantrieb 
durch Fahrbahnlinienerkennung eine vorgegebene Strecke fahren zu lassen. Als
technische Bestandteile wird ein Raspberry Pi 3 mit dazugehöriger Kamera
gewählt, dazu ein Sensor zur Abstandsmessung, um etwaige Kollisionen zu
vermeiden. Implementiert wird der Code in Python unter der Nutzung der
Bildverarbeitungslibrary openbCV. \\
Das Auto soll  sowohl ein 4m lange Strecke gradeaus als auch in Kurven von 90° 
zwischen zwei Fahrbahnlinien die Spur halten. Idealerweise ist es möglich, dass 
der Wagen auch mit Linienunterbrechungen weitersteuern und so eine acht-förmige 
Strecke abfahren kann wie sie beim Carolo-Cup, einem Wettbewerb für autonome
Modellfahrzeuge, befahren wird.
